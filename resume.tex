\documentclass{resume} % Use the custom resume.cls style
\usepackage{enumitem} % for item spacing
\usepackage{scalerel} % for inline images

\usepackage[left=0.4 in,top=0.4in,right=0.4 in,bottom=0.4in]{geometry} % Document margins
\newcommand{\tab}[1]{\hspace{.2667\textwidth}\rlap{#1}} 
\newcommand{\itab}[1]{\hspace{0em}\rlap{#1}}
\name{JOHN GIORSHEV} % Your name

\address{
    \scalerel*{\includegraphics{images/location.png}}{|} \href{https://www.google.com/maps/place/Greater+Toronto+Area}{Toronto, ON}\space\space\space\space\space\space
    \scalerel*{\includegraphics{images/email.png}}{|} \href{mailto:johngiorshev1@gmail.com}{johngiorshev1@gmail.com}\space\space\space\space\space
    \scalerel*{\includegraphics{images/phone.png}}{|} \href{tel:+17058164454}{705-816-4454}\space\space\space\space\space\space
    \scalerel*{\includegraphics{images/linkedin.png}}{|} \href{https://www.linkedin.com/in/johngiorshev/}{johngiorshev}
}  % Your phone number, email, linkedin, and (optional) website

\begin{document}

\begin{rSection}{EMPLOYMENT EXPERIENCE}
    \begin{itemize}
        \item {\bf Software Developer Intern} - SpaceRyde \hfill {Sep 2021 - Sep 2022}
        \begin{itemize}[topsep=-10pt]
            \setlength\itemsep{-0.35em}
            \item[\textbullet] {\bf Embedded Development} - ESP32 embedded development using PlatfromIO. MQTT.
            \item[\textbullet] {\bf User Interfaces} - Wrapped ROS2 components via QT GUIs and FSMs for human interface and control.
            \item[\textbullet] {\bf Telemetry} - Data acquisition via LabJack. Prioritization, packetization, and streaming of sensors and metrics. Downstream dashboards with TimescaleDB and Grafana. 
            \item[\textbullet] {\bf Simulator} - Decreased computation time of a 6DOF rigid body simulator by a factor of 50 using Eigen lib and multithreading.
            Integrated with ROS2 and GeographicLib.
        \end{itemize}
        \item {\bf Cyber Threat Intelligence Engineering Co-op} - Bell Canada Enterprises \hfill {Summer 2020}
        \begin{itemize}[topsep=-10pt]
            \setlength\itemsep{-0.35em}
            \item[\textbullet] Integrated a MISP server with an Elasticsearch database using RESTful APIs. Elastic Stack. Kafka.\\
            Improved Bell's security information infrastructure. Collaborated using Confluence and Jira.
            \item[\textbullet] Mattermost (slack) notification bot allowing faster team response to latest disseminated exploits.
        \end{itemize}
    \end{itemize}
\end{rSection}

\begin{rSection}{Projects}
    \begin{itemize}
        \item {\bf Choose} - \href{https://github.com/jagprog5/choose/}{jagprog5/choose}\\
        Created a stream manipulator that can be \textbf{significantly faster} than standard tools including GNU sed and GNU sort.
        Uses the PCRE2 regex C API for pattern matching and Ncurses C API for TUIs.
        \item {\bf Key Value} - \href{http://www.keyvalue.ca/}{keyvalue.ca}\\
        HTTPS enabled self-hosted web site for users to store key value pairs. Docker Compose. Rust HTTP backend integrated with SQLite. TLS termination with NGINX and Let's Encrypt for secure ingress.
        \item {\bf Terminal} - \href{https://github.com/jagprog5/terminal}{jagprog5/terminal}\\
        Psuedo-terminal interface using linux syscalls and SDL2 rendering. Supports UTF8 and ANSI escape codes.
        \item {\bf Sparse Distributed Representation} - \href{https://github.com/jagprog5/SDR/}{jagprog5/sdr}\\
        C++ sparse math library. Template metaprogramming and specializations for data structures. Profile guided optimization. Integrated with CMake, CI/CD: static analyzers, fuzzy testing for complete code coverage.
        \item {\bf AWS Log Uploader}\\
        Created a multithreaded utility that tracks a log folder and uploads to AWS CloudWatch using boto3. Robust to sudden interruption and prolonged internet outage.
        \item {\bf Data Analytics} - \href{https://github.com/jagprog5/resume/blob/main/data-analytics-projects.md}{jagprog5/data-analytics}\\
        Implemented multi-layer perceptrons using PyTorch and C99. Implemented hopfield networks.\\
        Used various sklearn models on Kaggle datasets. Created a shell wrapper of Matplotlib for data visualization. 
    \end{itemize}
\end{rSection}

\begin{rSection}{Design Teams}
    \begin{itemize}
        \item {\bf QMind} - Queen's AI/ML Software Team \hfill {Sep 2019 - May 2020}\\
        Presented at \href{https://medium.com/@cameronmackinn/cucai-2020-c821c15f3ab4}{CUCAI 2020}. Used OpenCV and Flask to create a computer vision API.
        \item {\bf Network Security Team} \hfill {Sep 2018 - May 2019}\\
        Penetration testing on virtual machines with Metasploit. Exploited apps in a sandboxed environment.
\end{itemize}
\end{rSection}

\begin{rSection}{EDUCATION}
    \begin{itemize}
        \item {\bf Computer Engineering} - Queen's University (First Class Honours) \hfill {Sep 2018 - May 2023}
    \end{itemize}
\end{rSection}

\end{document}
